\chapter{Studium przypadku (Marek Grudkowski)}

W ramach studium przypadku po zaimplementowaniu aplikacji zaprojektowano kurs o tematyce ekologicznej. W ramach wykorzystania technik grywalizacji zosta�a mu nadana fabu�a oraz cel, jaki potencjalny uczestnik musi osi�gn��. Opr�cz tego wykorzystano w nim wiele innych technik grywalizacji takich jak: bohater ludzko�ci, kamienie milowe, czy tortury przerwy. Kurs sk�ada si� z opisu fabularnego (odno�nik do rysunku) oraz zestawu 9 aktywno�ci podzielonych (odno�nik do tabeli) na nawyki, zadania otwarte oraz zadania zamkni�te. 

Opis fabularny kursu

Tabela z aktywno�ciami

Przy pr�bie wykorzystania kursu w celach edukacyjnych, zalecane jest jednak zwi�kszenie liczby aktywno�ci tak, by kurs mo�na by�o realizowa� przez okres minimum tygodnia, co da szanse na wyrobienie w uczestnikach rutyny, kt�ra zach�ci�aby ich do codziennej aktywno�ci na platformie oraz pozwoli�a wyrobi� nawyki zdefiniowane w kursie. Dzi�ki temu uczniowie mogliby �atwiej i lepiej opanowa� materia� ca�y czas odnosz�c wra�enie, �e  oddaj� si� rozrywce w postaci gry.

