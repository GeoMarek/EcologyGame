\chapter{Analiza}

\section{Wizja systemu}

\subsection{Interesariusze i ich problemy}

Minione kilkanaście lat to okres dynamicznego rozwoju Internetu i globalizacja
informacji. Wraz z jego rozwojem pojawiają coraz to nowsze sposoby na
wykorzystanie jego zasobów. Jednym z nich jest nauczanie na odległość.
Pozwala ono na zgłębianiu tajników różnych dziedzin bez potrzeby fizycznego
kontaktu z osobą przekazującą wiedzę.

Zdalne nauczanie stało się bardzo głośnym tematem w ciągu ostatniego roku,
czego przyczyną jest światowa pandemia korona wirusa. Wiele szkół zamknięto i
nakazano przejście na nauczanie zdalne. Ważnym pytaniem jest to, jak pandemia 
wpłynęła zarówno na rozwój zdalnego nauczania pod względem informatycznym, jak 
i na metody stosowane przez pedagogów, którzy gwałtownie z dnia na dzień musieli 
zmienić swoje codzienne nawyki pracy. Głównym problem związanym z nauczaniem 
zdalnym jest przełamanie barier edukacyjnych takich jak miejsce zamieszkania, 
sztywne pory zajęć czy umiejętność obsługi oprogramowania. W dużej mierze 
przełamanie owej bariery zależy w głównie od chęci zdobycia wiedzy, a uczniem 
może być zarówno dziecko, jak i osoba dorosła.

W okresie lockdownu wielu z uczniów szkół podstawowych, ponadpodstawowych, czy 
nawet studentów traciło zapał do nauki. Wielu nauczycieli twierdzi, że poprzez 
komputer trudno z uczeniem utworzyć więź. Okazuje się czymś, co zwiększa zaufanie 
i motywacje uczniów do nauki. Problemem podczas nauki zdalnej jest też 
nieprzystosowanie nauczycieli do prowadzenia  zajęć w formie zdalnej. Wielu z 
nich posiada wspaniałe materiały, ale mogą je wykorzystać tylko na zajęciach 
stacjonarnych. Brak odpowiednich umiejętności  technicznych oraz niska motywacja 
uczniów znacząco wpływają również na to, jakie podejście ma nauczyciel. Te dwie
grupy - uczniowie i nauczyciele są głównymi interesariuszami projektu.

Problemem u większości nauczycieli jest słaba znajomość oprogramowania 
wykorzystywanego podczas nauki zdalnej oraz brak skutecznych sposobów motywowania 
uczniów do chociażby podjęcia próby nauki. Uczniowie potrzebują właśnie mechanizmu, 
który wytworzył by w nich pozytywne nawyki oraz chęć zdobywania wiedzy. Tutaj 
świetnym rozwiązaniem może okazać proponowana przez nas aplikacja, wykorzystujące 
mechanizmy gamifikacji. 

\subsection{Proponowane rozwiązanie}

Proponowane rozwiązanie to aplikacja, na której nauczyciel będzie mógł  tworzyć i realizować kursy dla swoich podopiecznych. Kursy takie dodatkowo, by  zachęcić uczniów do udziału w nich będą opierać się na mechanizmach gamifikacji. 
W każdym kursie nauczyciel będzie mógł w prosty sposób tworzyć aktywności i monitorować, w których uczniowie będą mogli brać udział. Aktywności definiowane przez autora takiego kursu można podzielić na 3 grupy:
\begin{itemize}
\item zadania - zwykłe zadania, na które uczniowie przesyłają odpowiedzi, nauczyciel 
po sprawdzeniu wysłanego rozwiązania przydziela jej autorowi odpowiednią ilość punktów, 
jeśli zadanie takie jest obowiązkowe to może odejmować punkty za nie wysłanie odpowiedzi
\item wydarzenia - jest to aktywność grupowa, która ma zmobilizować uczniów do współpracy; 
nauczyciel definiuje tutaj zestaw pytań zamkniętych, a uczniowie muszą odpowiedzieć na 
nie określoną liczbę razy, dodatkowym aspektem jest to, że odpowiedzi można udzielać 
raz na jakiś czas
\item ciekawostki - nie można tego nazwać stricte aktywnością, z założenia mają to być 
notatki, przypomnienia, ciekawe fakty, które będą się wyświetlać losowo uczestnikom kursu 
na zasadzie podobnej do reklam
\end{itemize}
Dla nauczyciela ważne jest też monitorowanie tego, kto bierze udział w kursie i w 
jakim stopniu jest zaangażowany. Dlatego konieczne jest również wprowadzenie dla autora 
kursu panelu, który umożliwia mu przeglądanie takich informacji, czy generowanie raportów 
do szkolnej dokumentacji. 

Od strony ucznia aktywności wyglądają zupełnie inaczej. Otóż podczas dołączania do 
kursu każdy uczestnik otrzymuje swojego własnego bohatera, którego musi rozwijać 
poprzez wykonywanie misji. Jako misje rozumiemy tutaj właśnie zadania, które wcześniej 
zostały zdefiniowane przez nauczyciela. Aplikacja ma zapewnić od strony ucznia, by uważał 
on iż, wykonywanie tych zadań to nie wysłanie notatki z lekcji, lecz przykładowo walka z 
potworem, by awansować swoją postać na wyższy poziom. Dla ucznia cały kurs powinien posiadać 
taką otoczkę. Podczas rozwijania swojej postaci może przydzielać jej wybrane przez siebie 
atrybuty oraz kupować dla niej przedmioty, a wydarzenia specjalne mają zapewniać tak wysokie 
nagrody, by prosili prywatnie swoich kolegów i koleżanki z klasy, by pomogli im ją wygrać. 

\subsection{Wymagania funkcjonalne, jakościowe i ograniczenia}

Zarówno nauczyciele jako autorzy kursów, jak i uczniowie jako ich uczestnicy mają specyficzne
wymaganie dotyczące aplikacji. Podczas jej projektowania należy wziąć pod uwagę, to w jakim
kontekście będą z niej korzystać, co może sprawić problemy, a co może utrudnić ich
wystąpienie. Kluczowe w tym aspekcie są wymagania funkcjonalne. Są to takie wymagania, które dotyczą wyniku zachowania systemu, który powinien zostać dostarczony przez funkcję systemu. 

\begin{table}[h]
\caption{Wymagania funkcjonalne}
\label{chapter4_tab_funkcjonalne}
\centering
\begin{tabular}{|l|l|c|}
\hline
Użytkownik & Funkcja & Priorytet \\ \hline

\multirow{12}{*}{\makecell{Nauczyciel \\ (autor kursu)}} 

	& Możliwość tworzenia kursów & 1
		\\ \cline{2-3} 
	& Możliwość wyboru otoczki fabularnej & 3         		
		\\ \cline{2-3} 
    & Możliwość zapraszania użytkowników & 1          	
    	\\ \cline{2-3} 
	& Kontakt z uczestnikami kursu & 2   
		\\ \cline{2-3} 
	& Możliwość tworzenia zadań & 1
		\\ \cline{2-3} 
	& Możliwość zapisywania szablonów zadań & 2
		\\ \cline{2-3} 
	& Tworzenie wydarzeń dla wszystkich uczestników & 1
		\\ \cline{2-3} 
	& Funkcja do definiowania wyświetlanych uczestnikom notatek & 2
		\\ \cline{2-3} 
	& Dostęp do rozwiązań nadesłanych przez uczestników & 1
		\\ \cline{2-3} 
	& Podgląd panelu aktywności uczestników & 2
		\\ \cline{2-3} 
	& Możliwość indywidualnego kontaktu z uczestnikami & 3
		\\ \cline{2-3} 
	& Dostęp do raportów z przebiegu poszczególnych zadań & 2
		\\ \hline
	

\multirow{8}{*}{\makecell{Uczeń \\ (uczestnik kursu)}} 
	
	& Możliwość dołączania do kursów & 1
		\\ \cline{2-3} 
	& Możliwość tworzenia postaci per kurs & 1
		\\ \cline{2-3} 
	& Możliwość rozwijania postaci & 1
		\\ \cline{2-3}
	& Dostęp do aktywności, za które są nagrody & 1
		\\ \cline{2-3}
	& Dostęp do sklepu z przedmiotami dla postaci & 2
		\\ \cline{2-3}
	& Możliwość indywidualnego kontaktu z innymi uczestnikami & 3
		\\ \cline{2-3}
	& Dostęp do rankingu postaci w danym kursie & 3
		\\ \cline{2-3}
	& Dostęp do panelu z powiadomieniami & 2
		\\ \hline


\end{tabular}
\end{table}

Poza wymaganiami funkcjonalnymi w procesie projektowania należy również wziąć pod uwagę
wymagania jakościowe oraz ograniczenia, jakie mogą występować. Wymaganie jakościowe to wymaganie, które dotyczy jakości tworzonego oprogramowania, które nie zostało określone przez wymagania funkcjonalne \cite{wolskipro}. Ograniczenia również są również wymaganiami. Różnią się one tym od poprzednich, tym że po prostu ogranicza przestrzeń rozwiązań poza to, co zostało zdefiniowane w wymaganiach funkcjonalnych i jakościowych. 

Wymagania jakościowe zazwyczaj dzielone są na kilka podgrup. W projektowanej aplikacji podzielono je właśnie w ten sposób i wyodrębniono 6 podstawowych kategorii:
\begin{itemize}
\item wydajność - możliwość jednoczesnego korzystania przez 1000 użytkowników
\item dostępność - zakładając, że zarówno uczniowie, jak i nauczyciele mogą korzystać aplikacji wszędzie, powinna ona być dostępna co najmniej 20 godzin na dobę
\item ochrona - zapewnienie bezpieczeństwa danych osobowych uczestników kursu i podjętych przez nich w nim aktywności 
\item przenośność - dostępność aplikacji z poziomu przeglądarki, zarówno na komputerze, jak i urządzeniu mobilnym
\item elastyczność - na chwilę obecną nie przewiduje się gruntownych zmian w projekcie systemu, jednakowoż warto zostawić w nim furtki na przyszłość, które pozwolą na dodawanie nowych funkcjonalności takich jak np. system osiągnięć 
\item konfigurowalność - możliwość definiowania otoczki fabularnej w kursie
\end{itemize} 
\vspace{0.3cm}
Ograniczenia wyodrębnione podczas analizy projektu:
\begin{itemize}
\item czasowe - czas realizacji projektu to początek grudnia
\item konieczność działania w specyficznych warunkach - potrzebny jest dostęp do Internetu
\item określone formaty danych - pobieranie szablonów i raportów w określonych formatach danych (xls, csv, json)
\item dokumentacja - utworzenie wiki z dokumentacją i wskazówkami dla użytkowników
\item sposób wdrożenia - 2 tygodniowy okres próbny z jednym kursem o tematyce ekologii w klasie szkoły ponadpodstawowej
\end{itemize}
