\chapter*{Streszczenie}
\indent W poni�szej pracy opisano etapy budowania platformy edukacyjnej wykorzystuj�cej szeroko poj�te mechanizmy grywalizacji, nazywan� zamiennie gamifikacj�. W pierwszym rozdziale om�wiono wst�p i cel pracy. W drugim rozdziale przedstawiono g��wne mechanizmy grywalizacji oraz tematyk� kursu realizowanego w ramach studium przypadku. W trzecim rozdziale dokonano przegl�du istniej�cych rozwi�za� w obszarze zdalnego nauczania i wykorzystywania mechanizm�w gamifikacji. W kolejnym opisano analiz� projektowanego systemu. Zawarto w nim takie elementy jak wizja systemu, diagramy przypadk�w u�ycia i diagram klas. W pi�tym rozdziale opisano projekt systemu, czyli metodyk� pracy, architektur� ca�ego systemu oraz przedstawiono schemat bazy danych. W sz�stym i si�dmym rozdziale opisano szczeg�y implementacji wybranych fragment�w systemu oraz studium przypadku. W ko�cowym rozdziale przedstawiono wnioski i spostrze�enia w postaci podsumowania.\vspace{0.5cm}\newline

\noindent \textbf{S�owa kluczowe:} gamifikacja, nauczanie zdalne, diagramy UML, serwis internetowy\vspace{0.5cm}\newline
\noindent \textbf{Dziedzina nauki i techniki, zgodnie z wymogami OECD:} Informatyka