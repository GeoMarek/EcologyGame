\chapter*{Abstract}
The outbreak of the corona virus pandemic in 2019 prompted a shift to remote learning in most schools around the world. At that time, the demand for software increased drastically. However, the market lacked a solution that could motivate students to actively work and learn and develop the necessary habits for the pandemic. 

Therefore, the aim of this study is to design a system for remote learning with the ability to develop positive habits among its users and increase motivation to work or study. For this purpose, gamification mechanisms were used, which make the user feel as if he/she is participating in a game. As part of the effectiveness testing, a case study was also conducted, in which a sample course on eco-friendly topics was designed. 

The design work was based on software engineering techniques, such as the analysis of similar systems, developing a vision of the system or UML diagrams. During the implementation work DevOps techniques were used to ensure the workflow. Then the scope of work was divided into tasks related to business logic and tasks related to the user interface. The application uses the following frameworks: Django written in Python and React written in Javascript.

\textbf{Doda� wnioski i spostrze�enia}

\noindent \textbf{Keywords:} gamification, remote learning, ecology 
\vspace{0.5cm}\newline
\noindent \textbf{Field of science and technology, as required by the OECD: } Computer Science