\chapter*{Abstract}
\indent The outbreak of the corona virus pandemic caused a shift to remote learning in most schools around the world. At that time, the demand for remote learning software increased drastically. However, the market lacked a solution that could motivate students to actively work and learn and develop the necessary habits for the pandemic. 

Therefore, the aim of this work is to make an application that allows distance learning while developing habits using gamification mechanisms and creativity techniques. As a case study, a course with an environmental theme was designed for this purpose. The scope of work included learning about the problem domain, developing a vision for the system, documenting the analysis and design of the system, and implementing it. 

During the design work, methods such as system vision, use case diagrams and class diagrams were used. The implementation work was done in accordance with the DevOps culture techniques and using such technologies as version control system and Github Actions. The application was created using two web frameworks: Django written in Python, which is responsible for the business logic, and React written in Javascript, which is responsible for the user interface.  

The final result was a ready and working application with extensive documentation. As part of the work, a course on ecological issues was created, which can serve as a template when designing new courses. No major difficulties were encountered during the work on the project, and almost all use cases identified during the analysis were implemented. The only functionality that did not work in full was the performance of group tests among the course participants. 
\vspace{0.5cm}\newline
\noindent \textbf{Keywords:} gamification, remote learning, ecology 
\vspace{0.5cm}\newline
\noindent \textbf{Field of science and technology, as required by the OECD: } Computer Science