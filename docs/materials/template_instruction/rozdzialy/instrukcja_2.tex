\chapter{Uwagi od strony technicznej}

\section{Opcje}

\begin{enumerate}
 \item \verb|strict| -- domy�lnie, klasa stara si� jak naj�ci�lej wype�nia� zalecenia
 \item \verb|nostrict| -- drobne modyfikacje typograficzne
 \begin{itemize}
  \item zmniejszenie wci�cia akapitowego z 1.25cm na 1.5em
 \end{itemize}
\end{enumerate}

\section{Wymagane pakiety}
Lista pakiet�w, kt�re s� wymagane do kompilacji (wi�kszo�� z nich jest zapewne zainstalowana
domy�lnie)
\begin{enumerate}
  \item \verb|polski| -- polonizacja \TeX'a
  \item \verb|fontenc| -- kodowanie znak�w
  \item \verb|inputenc| -- kodowanie znak�w
  \item \verb|helvet| -- wybiera font podobny do Arial
  \item \verb|geometry| -- ustawienie margines�w
  \item \verb|indentfirst| -- wci�cie pierwszego akapitu
  \item \verb|fancyhdr| -- paginacja
  \item \verb|titlesec| -- tytularia
  \item \verb|titletoc| -- formatowanie spisu tre�ci
  \item \verb|enumitem| -- wyliczenia numerowane i nienumerowane
  \item \verb|amsmath,amssymb,amsthm| -- standardowe pakiety matematyczne
  \item \verb|graphicx| -- do��czanie obrazk�w
  \item \verb|subfig| -- wiele obrazk�w na jednym rysunku
  \item \verb|caption| -- format podpisu pod obrazkiem
  \item \verb|tikz| -- generowanie schemat�w blokowych i innych rysunk�w
  \item \verb|listings| -- umieszczanie listing�w kodu w pracy
\end{enumerate}

\section{Fonty}

Wymaganym fontem jest Arial. Poniewa� taki font nie jest �atwo dost�pny w \LaTeX'u wi�c korzystamy
z fonta zast�pczego w pakiecie \verb|helv|. Wymagany font matematyczny nie zosta� podany. U�ywamy
zatem fontu z pakietu \verb|mathpazo|.
\begin{verbatim}
\usepackage{helvet}
\usepackage{mathpazo}
\renewcommand{\familydefault}{\sfdefault}
\end{verbatim}

Inn� wersje fontu bezszeryfowego mo�na uzyska� poprzez zrezygnowanie z pakietu \verb|helv|.

Przyk�adowy: $\sin(x)+ay^2$.
\[
 \sin(x)+ay^2
\]
